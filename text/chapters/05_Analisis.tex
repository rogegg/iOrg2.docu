\chapter{Análisis}

Siguiendo las necesidades del cliente sintetizadas en los objetivos[\ref{sec:objetives}] iniciaremos el análisis de los requisitos de la aplicación. Dividiremos la fase de análisis en dos: análisis de los requisitos funcionales y no funcionales \cite{ing_software}.

\section{Análisis de los requisitos funcionales}

Los requisitos funcionales son las características requeridas del sistema que expresan una funcionalidad de este:


\begin{itemize}
	\item \textbf{RF-1.} 
	\item \textbf{RF-2.} 
	\item \textbf{RF-3.} 
	\item \textbf{RF-4.} 
	\item \textbf{RF-5.} 
	\item \textbf{RF-6.} 
	\item \textbf{RF-7.} 
	\item \textbf{RF-8.} 
		\begin{itemize}
			\item \textbf{RF-8.1.} 
			\item \textbf{RF-8.2.} 
			\item \textbf{RF-8.3.} 
			\item \textbf{RF-8.4.} 
			\item \textbf{RF-8.5.} 
		\end{itemize}
	\item \textbf{RF-9.} 
\end{itemize}


\section{Análisis de los requisitos no funcionales}

Los requisitos no funcionales hacen referencia a las cualidades del producto requeridas por el cliente. Estas características no se limitan a la aplicación, pueden ser referentes al sistema, al proceso de desarrollo o incluso al entorno.

\begin{itemize}
  \item \textbf{RNF-1.} 
  \item \textbf{RNF-2.} 
  \item \textbf{RNF-3.} 
  \item \textbf{RNF-4.} 
  \item \textbf{RNF-5.} 
  \item \textbf{RNF-6.} 
  \item \textbf{RNF-7.} 
  \item \textbf{RNF-8.} 
  \item \textbf{RNF-9.} 
\end{itemize}


\section{Casos de uso}

\subsection{Actores}

\subsection{Casos de uso}

\begin{itemize}
  \item \textbf{CU-1.} Añadir una conexión.
  \begin{itemize}
    \item \textbf{Actores:} Usuario.
    \item \textbf{Tipo:} Primario, esencial.
    \item \textbf{Referencias:}
    \item \textbf{Precondición:}
    \item \textbf{Postcondición:} La nueva conexión será añadida a la lista y guardada.
    \item \textbf{Autor:} \autor.
    \item \textbf{Versión:} 1.0.
    \item \textbf{Propósito:} Añadir una nueva conexión.
    \item \textbf{Resumen:} El usuario rellenará una serie de campos y marcará unas opciones para añadir una nueva conexión a la lista.
    \end{itemize}
    \begin{table}[!ht]
      \begin{center}
	\begin{tabular}{|l|l|l|l|}
	  \hline
	  \multicolumn{4}{|c|}{{\bf Curso normal}}
	  \\ \hline
	  \multicolumn{2}{|c|}{{\bf Actor}} & \multicolumn{2}{c|}{{\bf Sistema}}
	  \\ \hline
	  {\it 1} & 
	  \begin{tabular}[c]{@{}l@{}}
	    Usuario: Pulsa el botón para\\
	    añadir una nueva conexión.\\
	  \end{tabular} &
	  &
	  \\ \hline
	  &
	  &
	  {\it 2} &
	  \begin{tabular}[c]{@{}l@{}}
	    El sistema muestra el formulario\\
	    para añadir nuevas conexiones. \\
	  \end{tabular}
	  \\ \hline
	  {\it 3} & 
	  \begin{tabular}[c]{@{}l@{}}
	    Usuario: Rellena los campos del \\
	    formulario, marca las opciones  \\
	    y pulsa en el botón de añadir.   \\
	  \end{tabular} &
	  &
	  \\ \hline
	  &
	  &
	  {\it 4} &
	  \begin{tabular}[c]{@{}l@{}}
	    El sistema almacena la conexión\\
	    y la añade a la lista de conexiones.\\
	  \end{tabular}
	  \\ \hline
	\end{tabular}
	\caption{CU-1. Añadir nueva conexión.}
	\label{table:cu_1}
      \end{center}
    \end{table}
    \newpage