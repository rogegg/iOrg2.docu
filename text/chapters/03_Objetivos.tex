\chapter{Objetivos}

{\titulo} persigue el objetivo general de desarrollar una aplicación web de apoyo a la docencia que resuelva los inconvenientes ya descritos en iOrg1.0. Esta aplicación deberá obtener los datos de una hoja de cálculo de Google Drive en la cual el Departamento de Organización de Empresas tiene información docente de su asignatura y trabajar con estos datos para mostrarlos al alumno. Seguiremos la filosofía de \textbf{Software Libre} heredada del anterior proyecto.


\bigskip
A continuación vamos a listar los objetivos principales:

\begin{itemize}
  \item \textbf{OBJ-P-1.} Conseguir una primera versión funcional de la aplicación donde un alumno pueda consultar los conceptos y resolver algunas preguntas de la asignatura.
  \item \textbf{OBJ-P-2.} El proyecto debe estar estructurado de forma que sea escalable, permitiendo añadir en un futuro nuevas funcionalidades no dependientes de las actuales.
  \item \textbf{OBJ-P-3.} Conexión segura con Google Drive mediante su API.
  \item \textbf{OBJ-P-4.} Solventar los problemas de latencia al navegar entre secciones de la versión anterior.
  \item \textbf{OBJ-P-5.} Heredar el principio de Software Libre de la versión anterior, publicando este proyecto en GitHub y permitiendo a la comunidad participar en él.
\end{itemize}


\bigskip
También propondremos unos objetivos secundarios, que se intentarán alcanzar en la medida de lo posible:

\begin{itemize}
  \item \textbf{OBJ-S-1.} Añadir estadísticas básicas sobre la actividad de los alumnos en la aplicación.

  \item \textbf{OBJ-S-2.} Adaptar las vistas de la aplicación web a los tamaños de resolución de diferentes dispositivos.
  
  \item \textbf{OBJ-S-3.} Tener en cuenta el uso de la aplicación desde entornos con internet de velocidad limitada, mejorando así su usabilidad.

  \item \textbf{OBJ-S-4.} Despliegue de la aplicación en PaaS\footnote{``Platform as a Service''. Plataforma y entorno que permite a los desarrolladores crear y ejecutar aplicaciones totalmente en la nube}.
  \item \textbf{OBJ-S-5.} Gestión de los usuarios y sus roles, al menos permisos de alumnos y profesores.
  \item \textbf{OBJ-S-6.} Consulta, por parte de los profesores, de las estadísticas generadas por los alumnos, siendo estas útiles para valorar la actividad de los alumnos en la aplicación.
\end{itemize}



\bigskip
Para el desarrollo de este proyecto vamos a hacer uso de los conocimientos adquiridos en las siguientes asignaturas del {\grado}:

\begin{itemize}
  \item \textbf{Fundamentos de programación}, base  para cualquier desarrollo.
  \item \textbf{Programación orientada a objetos}, para entender este paradigma de programación.
  \item \textbf{Ingeniería del software}, para hacer un correcto análisis y planificación de un proyecto.
  \item \textbf{Base de datos}, para comprender la gestión de las bases de datos necesarias en este proyecto. 
  \item \textbf{Diseño de Aplicaciones para Internet}, donde vimos una breve introducción a Python, Django y el uso de varias APIs de Google. 
  \item \textbf{Infraestructura virtual}, para comprender el uso de sistemas PaaS, entornos virtuales, despliegue de aplicaciones y Git como sistema de control de versiones.
\end{itemize}


\section{Repercusión de los objetivos}
El desarrollo de este proyecto debería resolver los objetivos principales para crear una base de proyecto en el cual se podrá seguir trabajando en un futuro y crear un apoyo docente para las asignaturas de el Departamento de Organización de Empresas. Con estos objetivos se crea una base de un desarrollo escalable, en el que se podrán incluir sin problemas nuevas funcionalidades completando cada vez más este proyecto.

\bigskip
Los objetivos secundarios darán más consistencia al proyecto mejorando su rendimiento, expansión y usabilidad, así como un valor añadido para las asignaturas del departamento, facilitando la labor de docentes y el aprendizaje de los alumnos. Se podrá también tener un primer acercamiento por parte de la comunidad docente al desplegarla en un PaaS y estudiar las carencias y virtudes tras su uso.


%\section{Conexión entre los objetivos}
