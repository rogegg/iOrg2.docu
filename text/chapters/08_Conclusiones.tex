\chapter{Conclusiones y vías futuras}

\section{Conclusiones}
\bigskip
	A lo largo del {\grado} hemos realizado distintas prácticas centradas en distintos ámbitos de los proyectos informáticos pero, por motivos obvios de planificación de tiempos, no hemos podido desarrollar un proyecto desde su inicio, reuniones con el ``cliente'', hasta su fin con la entrega de este.  Durante estos meses que ha durado el desarrollo del Trabajo Fin de Grado hemos podido experimentar el ciclo completo de un proyecto en sus diferentes fases. 

\bigskip
	Cuando, durante estos años, nos hablaban de la importacia de la Ingeniería del Software en los proyectos, hacíamos análisis de diferentes escenarios simulados o de ejemplos de proyectos que no íbamos a desarrollar, imaginábamos la importancia de estos análisis pero quizá no llegásemos a comprender totalmente la necesidad de trabajar a conciencia en estas fases del desarrollo. {\titulo} nos ha permitido trabajar con un cliente, ver sus necesidades y realizar un análisis de estas, diseñar  y desarrollar una aplicación donde, aunque tutorizada en muchas fases, hemos podido experimentar un trabajo no tan arropado por el ámbito académico. Esto nos ha servido para acercarnos a un entorno profesional con unos objetivos, unos plazos definidos y un proyecto relativamente amplio.

\bigskip
	Aunque ya hablábamos de la importancia de las primeras fases de la aplicación, donde se analiza y diseña la solución a implementar hemos conocido y experimentado en primera persona estas recomendaciones. Un análisis exhaustivo y detallado da pie a un cuidadoso diseño para cumplir los objetivos del cliente. Tras esto, la implementación debería ceñirse y desarrollarse sin mayor problema. Nuestra experiencia en proyectos amplios es muy escasa, por lo que algunas decisiones del análisis y diseño nos han condicionado a buscar soluciones en la implementación que creemos podrían mejorarse. Es aquí cuando hemos podido observar que realmente un diseño pobre ocasiona grandes pérdidas de recursos en un proyecto.

\section{Vías futuras}	

\bigskip
{\titulo} se ha desarrollado sobre una plataforma y arquitectura estable y en gran medida escalable lo que creemos le ofrece un potencial de continuar el proyecto. A su vez está desarrollada sobre una licencia que permite la continuación del proyecto por otros autores bajo los mismo términos que esta se creó y apoyada en la plataforma GitHub para interactuar con su comunidad.

\bigskip
Durante el desarrollo se han ido ocurriendo posibles caminos futuros para la aplicación. Estas mejoras no se contemplan en este proyecto y se proponen para futuras versiones del mismo:

 \begin{itemize}
    \item \textbf{Permisos en la hoja de Google Drive.} Actualmente los permisos de edición se gestionan desde la hoja de cálculo de Google Drive. Se propone poder asignar permisos de edición al usuario profesor desde la aplicación mediante la API de Google Drive.

    \item \textbf{Historial de preguntas y exámenes por usuario.} La aplicación no se centra en que los profesores puedan consultar los ejercicios realizados por los alumnos pero quizá en un futuro necesite cambiar y almacenar en un rango de tiempo las actividades concretas más allá de sólo almacenar los resultados.

    \item \textbf{Exportar resultados a pdf.} Si la aplicación sirve realmente de apoyo al profesorado como medio de consulta de actividades de los alumnos puede ser útil que estos puedan exportar los resultados a un fichero pdf.

    \item \textbf{Mejora de la interfaz.} Aunque hemos presentado una interfaz clara y sencilla, con el uso de la plataforma por parte de los usuarios se obtienen opiniones y muchas de ellas se centran en la usabilidad. Un ejercicio útil sería estudiar estos casos y adecuar la interfaz al usuario final.

    \item \textbf{Prescindir de la hoja de Google Drive.} Actualmente los datos de la asignatura se editan en una hoja de Google Drive y aunque se ha aprendido del uso de esta API de Google creemos que para mejorar la escalabilidad de la plataforma esta deberá gestionar directamente los datos desde sus propios formularios.
 \end{itemize}


\bigskip
	Ya desarrollada la aplicación web y aunque se la ha capacitado de una adaptación a múltiples resoluciones de pantalla para que pueda consultarse desde diversos dispositivos, no olvidamos la vía de desarrollo de una aplicación móvil que complemente el proyecto. Una aplicación nativa podría proporcionar funcionalidades adicionales, la más importante que se propone es la opción de poder realizar ciertas actividades sin conexión a internet y ofrecer las actualizaciones al conectarse.