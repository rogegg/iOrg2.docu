\chapter{Resumen}

\section{Breve resumen y palabras clave}
\noindent{\textbf{Palabras clave}: \textit{\keywords}}\\


\bigskip
{\titulo} es un proyecto que nace de la necesidad en el Departamento de Organización de Empresas de la Facultad de Ciencias Económicas y Empresariales de la Universidad de Granada de desarrollar una aplicación móvil como apoyo a la docencia de sus asignaturas.

\bigskip
Así pues, se inició una colaboración con la Oficina de Sofware Libre para integrar las nuevas tecnologías en la docencia de estas asignaturas. Inicialmente se desarrolló como una aplicación móvil híbrida\footnote{Combinación de tecnologías web que no son aplicaciones móviles nativas ni tampoco están basadas en Web, porque se empaquetan como aplicaciones para distribución y tienen acceso a las APIs nativas del dispositivo.} dado que no era posible, por parte del departamento, mantener un servidor back-end y una base de datos.

\bigskip
{\titulo} pretende continuar con la idea original de explotar las virtudes de las nuevas tecnologías en el ámbito docente y a su vez solventar y mejorar situaciones de la primera versión híbrida, soluciones y decisiones que desarrollaremos en el siguiente capítulo. Sin embargo en esta versión se deja a un lado la aplicación móvil y se apuesta por el análisis y desarrollo de una solución  que pasa por un servidor web.

\bigskip
%Qué pretende de cara al alumno.

\bigskip
%Qué pretende de cara a los docentes.




\newpage
\begin{center}
{\LARGE\bfseries\tituloEng: }{\LARGE\bfseries\subtitulo}\\
\end{center}
\begin{center}
\autor\
\end{center}

\section{Extended abstract and key words}

\noindent{\textbf{Key words}:\textit{\keywords}}\\
