\chapter{Resumen}

\section{Breve resumen y palabras clave}
\noindent{\textbf{Palabras clave}: \textit{\keywords}\\


\bigskip
{\titulo} es un proyecto que nace a partir del trabajo realizado como becario ICARO\footnote{El Portal de Gestión de Prácticas en Empresa y Empleo utilizado por las Universidades Públicas Andaluzas, la Universidad Politécnica de Cartagena y la Universidad Autónoma de Madrid.} en el Departamento de Organización de Empresas de la Facultad de Ciencias Económicas y Empresariales de la Universidad de Granada y en colaboración directa con la Oficina de Sofware Libre. En dicho proyecto el Departamento de Organización de Empresas necesitaba desarrollar una aplicación móvil como apoyo a la docencia de sus asignaturas. 

\bigskip
Con el objetivo de integrar las nuevas tecnologías en la docencia de estas asignaturas nace iOrg1.0. Esta aplicación se desarrolla como una aplicación móvil híbrida\footnote{Combinación de tecnologías web como basadas generalmente en Javascript, HTML y CSS que no son aplicaciones móviles nativas, porque consisten en un WebView ejecutado dentro de un contenedor nativo, ni tampoco están basadas en Web, porque se empaquetan como aplicaciones para distribución y tienen acceso a las APIs nativas del dispositivo.} dado el poco período de tiempo del que se disponía y los pocos recursos capaces de sostener. No era posible, por parte del departamento, mantener un servidor back-end y una base de datos, por lo que se decidió para una primera versión de la aplicación, desarrollar el proyecto en un entorno de aplicación móvil híbrida, que accede a Google Drive para servir los datos directamente sin almacenarlos.

\bigskip
{\titulo} pretende continuar con la idea original de explotar las virtudes de las nuevas tecnologías en el ámbito docente y a su vez solventar y mejorar situaciones de la primera versión híbrida. Soluciones y decisiones que desarrollaremos en el siguiente capítulo.

\newpage
\begin{center}
{\LARGE\bfseries\tituloEng}\\
\end{center}
\begin{center}
\autor\
\end{center}

\section{Extended abstract and key words}

\noindent{\textbf{Key words}:\\
